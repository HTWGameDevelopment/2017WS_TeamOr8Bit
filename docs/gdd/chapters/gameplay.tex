\part{Gameplay and Mechanics}
\section{Gameplay}
The most basic premise is this:

\begin{itemize}
\item Two players (A.I., local or network) play in a hexagonal map.
\item Each player has a set of units and buildings.
\item Symmetry depends on the map being played.
\item The goal of each player is to destroy or conquer every enemy base.
\item During the match, each player takes turns where they can move their units, start attacks, take over buildings or build new ones.
\end{itemize}
\section{Mechanics}
\subsection{Moving units} %TODO
The core game mechanic of this game are moveable units which occupy empty tiles on our hexagonal board.
The units have the following parameters:

\begin{itemize}
    \item The faction $f$ to which the unit belongs.
    \item The number of health points $h$.
    \item A level $l$ indicating unit experience (currently unused).
    \item A dynamic defense point metric $d$ indicating defensive capabilities.
    \item A dynamic attack point metric $a$ indicating offensive capabilities.
    \item The maximum travel distance per turn $dpt$.
    \item A relation $D: terrain \to dpt$ for calculating the maximum travel distance over different terrain.
    \item The maximum attack range $ar$.
    \item A relation $A: terrain \to ar$ for calculating the maximum attack range over different terrain.
    \item Additional data for unit actions outside of simple attacks.
\end{itemize}
\subsection{Attacking}
The attacking system is deterministic without any random interference. The algorithm for determining the outcome
of an encounter is also easy to learn but hard to master.

\begin{align}
    adj &= a_{att} - (\begin{Vmatrix}p_{att} - p_{def}\end{Vmatrix} - \sum_{t=p_{att}}^{p_{def}} A_{att}(t))) \\
    d &= d_{def} - adj \\
    h_{def} &= h_{def} + d \\
    l_{def} &= l_{def} - d \\
    l_{att} &= l_{att} + d
\end{align}

%TODO graphs
\subsection{Defending} %TODO ref
A unit can go into defense mode which boosts the defensive metric $d$ by $1.5$ but renders the unit unable to move.
Going into and out of DM takes up one turn.
\subsection{Building units} %TODO ref
Units can be built in static and indestructable (but not invincible) bases.
A base can be tasked to continuously produce units for the owning player. Each base can only build one unit
at a time. The type of unit is determined by the player but not all bases can build all unit types. The length and cooldown
of each production also varies between unit types.
\subsection{Buildings} %TODO ref
\subsubsection{Trench}
Infantry units have a special action for building trenches. A trench is a terrain modifier which boosts the
defensive metric $d$ of infantry units by $2$ and increases the travel distance for infantry units along the trench.

Additionally some unit types can be ,,upgraded'' into a building with a significantly higher $d$. The original unit is replaced by the
building and that building cannot be ,,downgraded'' back.
\subsection{Special buildings} %TODO ref
Some buildings have special features that can be triggered under certain conditions. See level for details
,,levolution''
\section{Screen Flow} %TODO
There are four screens:

\begin{enumerate}
    \item Main Menu
    \item Level Creation
    \item Level
    \item Options Menu
\end{enumerate}
\section{Game Options} %TODO
The match can be modified with the following settings:

\begin{description}
    \item[difficulty] changes the difficulty when playing against an AI.
    \item[map size] changes the map size.
    \item[Various toggles] for enabling/disabling/breaking selected gameplay elements.
    \item[Max Moves Per Turn] optional. Only allow a fixed number of actions per turn.
\end{description}
\section{Saving} %TODO
Each match can be saved into a custom file format, loaded into the game and played.
\section{Cheats and Easter Eggs} %TODO
It wouldn't be much of an easter egg if we told you.
% anno cheat RUAMZUZLA
