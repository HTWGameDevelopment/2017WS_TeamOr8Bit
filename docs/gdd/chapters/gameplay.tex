\part{Gameplay and Mechanics}
\section{Gameplay}
The most basic premise is this:

\begin{itemize}
\item Two players (A.I., local or network) play on a hexagonal map.
\item Each player has a set of units \index{unit} and buildings \index{building}.
\item Each unit \index{unit} occupies one  tile \index{tile}. Each building \index{building} one or more.
\item Symmetry depends on the map being played.
\item The goal of each player is to destroy or conquer every enemy base. Campaign goals may vary.
\item During the match, each player takes turns where they can move their units \index{unit}, start attacks, take over buildings \index{building} or build new ones.
\end{itemize}
\subsection{Round setup}
Players take turns (rounds). Actions are performed immediately and cannot be undone. The round ends by clicking a button on screen.
\section{Mechanics}
\subsection{Moving units \index{unit}}
The core game mechanic of this game are moveable units \index{unit} which occupy empty  tile \index{tile} on our hexagonal  board \index{board}.
The units \index{unit} have the following parameters:

\begin{itemize}
    \item The faction $f$ to which the unit \index{unit} belongs.
    \item The position $p$.
    \item The number of health points $h$.
    \item A level $l$ indicating unit \index{unit} experience (currently unused).
    \item A dynamic defense point metric $d$ indicating defensive capabilities.
    \item A dynamic attack point metric $a$ indicating offensive capabilities.
    \item The maximum travel distance per turn $dpt$.
    \item A relation $T: terrain \to dpt$ for calculating the maximum travel distance over different terrain.
    \item The maximum attack range $ar$.
    \item A relation $A: terrain \to ar$ for calculating the maximum attack range over different terrain.
    \item A relation $D: terrain \to ar$ for calculating the attack damage over different terrain.
    \item The maximum viewing range $vr$.
    \item A relation $V: terrain \to vr$ for calculating the visibility over different terrain.
    \item Additional data for unit \index{unit} actions outside of simple attacks.
\end{itemize}
\subsection{Moving}
Every unit type has a maximum travel distance and a different movement ,,cost'' per terrain type.
\subsection{Visibility}
Every unit type has a maximum viewing distance and a different viewing ,,cost'' per terrain type.
\subsection{Attacking}
Every unit type has a maximum attack range and a different attack range ,,cost'' per terrain type.
The attacking system is deterministic without any random interference. The algorithm for determining the outcome
of an encounter is also easy to learn but hard to master.

\begin{align}
    \text{attack possible?} &= (\begin{Vmatrix}p_{att} - p_{def}\end{Vmatrix} - \sum_{t=p_{att}}^{p_{def}} A_{att}(t)) > 0 \\
    adj &= a_{att} - \sum_{t=p_{att}}^{p_{def}} D_{att}(t) \\
    d &= d_{def} - adj \\
    h_{def} &= h_{def} + d \\
    l_{def} &= l_{def} + adj \\
    l_{att} &= l_{att} + d
\end{align}

$a$ is static but may change during balancing. $l$ is unused in the current algorithm.
\subsection{Defending}
A unit \index{unit} can go into defense mode which boosts the defensive metric $d$ by $1.5$ but renders the unit \index{unit} unable to move.
Going into and out of DM takes up one turn.
\subsection{Building units \index{unit}}
\label{building units}
Units can be built in static and indestructable (but not invincible) bases.
A base can be tasked to continuously produce units \index{unit} for the owning player. Each base can only build one unit \index{unit}
at a time. The type of unit \index{unit} is determined by the player but not all bases can build all unit \index{unit} types. The length and cooldown
of each production also varies between unit \index{unit} types.
\subsection{Buildings}
\subsubsection{Production Base}
Base type which produces units \index{unit} for the owning player. See~\ref{building units} for details.
\subsubsection{Radar Base}
Detects player movement of large units in a larger radius.
\subsubsection{Stronghold}
Simple camping spot with a defensive boost.
\subsubsection{Trench}
Infantry units \index{unit} have a special action for building \index{building} trenches. A trench is a terrain modifier which boosts the
defensive metric $d$ of infantry units \index{unit} by $2$ and increases the travel distance for infantry units \index{unit} along the trench.
\subsection{Tower}
Additionally some unit \index{unit} types can be ,,upgraded'' into a building \index{building} with a significantly higher $d$. The original unit \index{unit} is replaced by the
building and that building \index{building} cannot be ,,downgraded'' back.
\subsection{Special buildings \index{building}} %TODO ref
Some buildings \index{building} have special features that can be triggered under certain conditions. See~\ref{level} for details.
\section{Screen Flow} %TODO
There are four screens:

\begin{enumerate}
    \item Main Menu
    \item Level Creation
    \item Level
    \item Options Menu
\end{enumerate}
\section{Game Options} %TODO
The match can be modified with the following settings:

\begin{description}
    \item[difficulty] changes the difficulty when playing against an AI (if we have time to make one).
    \item[map] changes the map.
    \item[Various toggles] for enabling/disabling/breaking selected gameplay elements.
    \item[Max Moves Per Turn] optional. Only allow a fixed number of actions per turn.
\end{description}
\section{Saving}
Each match can be saved into a custom file format and loaded back into the game.
\section{Cheats and Easter Eggs}
There might be cheats.
% anno cheat RUAMZUZLA
